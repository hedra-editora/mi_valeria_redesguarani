\textbf{Nas redes guarani} reúne textos de autores guarani, de antropólogos vinculados ao Centro de Estudos Ameríndios da Universidade de São Paulo (\textsc{ce}st\textsc{a}--\textsc{usp}) e de pesquisadores convidados de outras instituições. A coletânea se volta para redes guarani de pessoas, lugares e práticas de conhecimentos, que trazem consigo singularidades dentro da própria multiplicidade de coletivos, como Mbya, Ava, Nhandeva, Xiripa, Tupi, Tupi"-Guarani, Kaiowá, Pai Tavyterã, entre outros. 

\textbf{Dominique Tilkin Gallois} é docente no Departamento de Antropologia Social da Universidade de São Paulo e pesquisadora no \textsc{ce}st\textsc{a}-\textsc{usp}.

\textbf{Valéria Macedo} é docente na área de antropologia do Departamento de Ciências Sociais da Escola de Filosofia, Letras e Ciências Humanas da Universidade Federal de São Paulo (Unifesp) e pesquisadora colaboradora do \textsc{ce}st\textsc{a}--\textsc{usp}.

\textbf{Coleção Mundo Indígena} reúne materiais produzidos com pensadores de diferentes povos indígenas e pessoas que pesquisam, trabalham ou lutam pela garantia de seus direitos. Os livros foram feitos para serem utilizados pelas comunidades envolvidas na sua produção, e por isso uma parte significativa das obras é bilíngue. Esperamos divulgar a imensa diversidade linguística dos povos indígenas no Brasil, que compreende mais de 150 línguas pertencentes a mais de trinta famílias linguísticas.





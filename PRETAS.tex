\textbf{Nas Redes Guarani} busca seguir as múltiplas conexões de pessoas e de conhecimentos entre
os muitos povos reunidos sob o nome Guarani, partindo do cruzamento de muitas linhas, muitas
caminhadas e experiências, que os nomes — a começar por Guarani — tentam mapear sempre de modo
incompleto. A publicação reúne textos de antropólogos vinculados ao Centro de Estudos Ameríndios
da Universidade de São Paulo (\versal{CE}st\versal{A}-\versal{USP}) que trabalham com populações
guarani, de pesquisadores convidados de outras instituições, além de professores, cineastas e
líderes guarani.

\textbf{Dominique Tilkin Gallois} é docente na área de antropologia do Departamento de Ciências
Sociais da Universidade Federal de São Paulo e pesquisadora colaboradora no Centro de Estudos
Ameríndios da Universidade de São Paulo. Possui graduação em Sciences Sociales Economiques et
Politiques pela Université Libre de Bruxelles e mestrado e doutorado em Ciência Social
(Antropologia Social) pela Universidade de São Paulo.

\textbf{Valéria Macedo} é docente do Departamento de Antropologia da Faculdade de Filosofia,
Letras e Ciências Humanas e Coordenadora do Centro de Estudos Ameríndios da Universidade de São
Paulo. Fez graduação em Ciências Sociais na \versal{USP} e Cinema na \versal{FAAP}, e concluiu o
mestrado e o doutorado no Programa de Pós-Graduação em Antropologia Social da \versal{USP}.

\textbf{Mundo Indígena}, coleção da Editora Hedra, reúne de um lado as cosmologias, histórias do
mundo e de aprendizado de culturas indígenas contadas por seus próprios interlocutores, e de outro
coletâneas e trabalhos acadêmicos dos grandes estudiosos da questão indígena no Brasil, de forma a
dar voz por vários meios à miríade de nações, etnias e culturas que tentam tentam reafirmar sua
existência e relevância enquanto povos originários.
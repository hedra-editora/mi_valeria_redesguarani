\textbf{Nas Redes Guarani} reúne textos de autores guarani, de antropólogos vinculados ao Centro de Estudos Ameríndios da Universidade de São Paulo (\versal{CE}st\versal{A}-\versal{USP}) e de pesquisadores convidados de outras instituições. A coletânea se volta para redes guarani de pessoas, lugares e práticas de conhecimentos, bem como se reconhece como parte delas.

\textbf{Dominique Tilkin Gallois} é docente no Departamento de Antropologia Social da Universidade de São Paulo e pesquisadora no \versal{CE}st\versal{A}-\versal{USP}.

\textbf{Valéria Macedo} é docente na área de antropologia do Departamento de Ciências Sociais da Escola de Filosofia, Letras e Ciências Humanas da Universidade Federal de São Paulo (Unifesp) e pesquisadora colaboradora do \versal{CE}st\versal{A}-\versal{USP}.

\textbf{Mundo Indígena}, coleção da Editora Hedra, reúne, de um lado, as cosmologias, histórias
e reflexões de povos indígenas nas palavras de seus próprios pensadores e, de outro, coletâneas
e trabalhos acadêmicos de grandes estudiosos da questão indígena no Brasil, reafirmando assim sua
existência e relevância em seus próprios termos.\par